\documentclass{article}
\usepackage[utf8]{inputenc}
\usepackage[german]{babel}
\usepackage{graphicx}
\usepackage[a4paper, total={6in, 10in}]{geometry}
\usepackage{listings}
\usepackage{color}

\title{Objektorientierte Programmierung: Aufgabenblatt 4}
\author{Florian Ludewig (185722)}
\date\today


\definecolor{my_red}{RGB}{216,60,104}
\definecolor{my_green}{RGB}{68,189,77}
\definecolor{my_grey}{RGB}{78,90,107}
\lstdefinestyle{customCStyle}{
  language=Java,
  numbers=left,
  stepnumber=1,
  breaklines=true,
  showstringspaces=false,
  keywordstyle=\color{my_red},
  stringstyle=\color{my_green},
  commentstyle=\color{my_grey},
  morecomment=[l][\color{magenta}]
}
\lstset{basicstyle=\ttfamily\small,style=customCStyle}

\begin{document}
\maketitle

\section*{Aufgabe 1}
\subsubsection*{Java Implementierung}
\texttt{Einwohner.java}
\begin{lstlisting}
public class Einwohner {
  protected double steuer_faktor = 0.1;
  protected int einkommen;
  protected int mindestAbgabe = 1;

  int zuVersteuerndesEinkommen() {
    return this.einkommen;
  }

  int steuer() {
    int abgabe = (int) Math.floor(this.zuVersteuerndesEinkommen() * this.steuer_faktor);
    return abgabe > this.mindestAbgabe ? abgabe : this.mindestAbgabe;
  }

  void setEinkommen(int einkommen) {
    this.einkommen = einkommen;
  }
}
\end{lstlisting}
\texttt{Adel.java}
\begin{lstlisting}
public class Adel extends Einwohner {
  Adel() {
    this.mindestAbgabe = 20;
  }
}
\end{lstlisting}
\texttt{Koenig.java}
\begin{lstlisting}
public class Koenig extends Einwohner {
  int steuer() {
    /*
      * Ich gehe davon aus, dass der Koenig auch den Mindestbetrag von 1 Gulden nicht
      * zahlen muss (ist aus 3. nicht genau hervorgegangen)
      */
    return 0;
  }
}
\end{lstlisting}
\texttt{Leibeigener.java}
\begin{lstlisting}
public class Leibeigener extends Bauer {
  int zuVersteuerndesEinkommen() {
    return this.einkommen > 12 ? this.einkommen - 12 : 12;
  }
}
\end{lstlisting}

\subsubsection*{Ergänztes Klassdendiagramm}
\includegraphics[width=\textwidth]{oop4_1.png}


\vspace*{2\baselineskip}
\section*{Aufgabe 2}




\end{document}