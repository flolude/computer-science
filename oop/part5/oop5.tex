\documentclass{article}
\usepackage[utf8]{inputenc}
\usepackage[german]{babel}
\usepackage[a4paper, total={6in, 10in}]{geometry}
\usepackage{listings}
\usepackage{color}

\title{Objektorientierte Programmierung: Aufgabenblatt 5}
\author{Florian Ludewig (185722)}
\date\today


\definecolor{my_red}{RGB}{216,60,104}
\definecolor{my_green}{RGB}{68,189,77}
\definecolor{my_grey}{RGB}{78,90,107}
\lstdefinestyle{customCStyle}{
  language=Java,
  numbers=left,
  stepnumber=1,
  breaklines=true,
  showstringspaces=false,
  keywordstyle=\color{my_red},
  stringstyle=\color{my_green},
  commentstyle=\color{my_grey},
  morecomment=[l][\color{magenta}]
}
\lstset{basicstyle=\ttfamily\small,style=customCStyle}

\begin{document}
\maketitle

\section*{Aufgabe 1}
\texttt{Auto.java}
\begin{lstlisting}
class Auto {
  private String kennzeichen = "J-AA 01";
  private int kilometerstand = 0;
  private ESitzplatz sitzplaetze = ESitzplatz.FUENF;
  private boolean antenneEingefahren = true;

  Auto() {
  }

  Auto(String knz) {
    kennzeichen = knz;
  }

  Auto(ESitzplatz sp) {
    sitzplaetze = sp;
  }

  public String getKennzeichen() {
    return kennzeichen;
  }

  public int getKilometerstand() {
    return kilometerstand;
  }

  public int getSitzplaetze() {
    return sitzplaetze == ESitzplatz.FUENF ? 5 : 2;
  }

  public void fahre(int kilometer) {
    System.out.println("Das Auto ist " + kilometer + " km gefahren");
    kilometerstand += kilometer;
  }

  public void fahreAntenneAus() {
    antenneEingefahren = false;
    System.out.println("Antenne wurde ausgefahren");
  }

  public void fahreAntenneEin() {
    antenneEingefahren = true;
    System.out.println("Antenne wurde eingefahren");
  }

  public void bereiteWaschenVor() {
    fahreAntenneEin();
  }

  public void wasche() {
    bereiteWaschenVor();
    System.out.println("Das Auto wird geschwaschen");
  }

  public String toString() {
    String antennenString = antenneEingefahren ? "eingefahren" : "ausgefahren";
    return "Das Auto mit dem Kennzeichen " + getKennzeichen() + " hat " + getSitzplaetze()
        + " Sitzplaetze und der Kilometerstand ist " + getKilometerstand() + " km. Die Antenne ist" + antennenString
        + ".";
  }
}
\end{lstlisting}
\texttt{ESitzplatz.java}
\begin{lstlisting}
public enum ESitzplatz {
  ZWEI, FUENF
}
\end{lstlisting}
\texttt{PickUp.java}
\begin{lstlisting}
class PickUp extends Auto {
  private int ladung = 0;
  private int f;

  int getLadung() {
    return ladung;
  }

  PickUp(int fassungsvermoegen) {
    super(ESitzplatz.ZWEI);
    f = fassungsvermoegen;
  }

  boolean beladen(int ldg) {
    if (ladung + ldg > f)
      return false;
    ladung += ldg;
    System.out.println("Du hast " + ldg + " Liter auf den PickUp geladen");
    return true;
  }

  void entladen(int ldg) {
    if (ldg > ladung)
      throw new Error("Du hast nur " + ladung + " Liter auf dem PickUp");
    ladung -= ldg;
    System.out.println("Du hast " + ldg + " Liter vom PickUp geladen");
  }

  void entladen() {
    entladen(ladung);
  }

  public void bereiteWaschenVor() {
    super.bereiteWaschenVor();
    entladen();
  }

  public String toString() {
    return super.toString() + " Der PickUp hat eine Ladefleache von " + f + " Liter und momentan sind " + ladung
        + " Liter aufgeladen.";
  }
}
\end{lstlisting}
\texttt{AutoTest.java}
\begin{lstlisting}
import java.util.ArrayList;

class AutoTest {
  ArrayList<Auto> list = new ArrayList<Auto>();

  AutoTest() {
    list.add(new Auto());
    list.add(new Auto(ESitzplatz.ZWEI));
    list.add(new Auto("X-YZ-42"));
    list.add(new PickUp(777));
    list.add(new PickUp(1234));

    printList();

    Auto a1 = list.get(0);
    a1.fahre(100);
    a1.fahreAntenneEin();

    Auto a2 = list.get(1);
    a2.fahre(200);
    a2.fahreAntenneAus();

    Auto a3 = list.get(2);
    a3.fahre(300);
    a3.wasche();

    PickUp p1 = (PickUp) list.get(3);
    p1.fahre(400);
    p1.beladen(500);
    p1.entladen(100);
    p1.bereiteWaschenVor();

    PickUp p2 = (PickUp) list.get(4);
    p2.fahre(600);
    p2.beladen(700);
    p2.entladen();

    printList();
  }

  private void printList() {
    for (Auto auto : list) {
      System.out.println(auto.toString());
    }
  }
}
\end{lstlisting}


\end{document}