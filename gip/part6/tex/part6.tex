\documentclass{article}
\usepackage[T1]{fontenc}
\usepackage{titlesec}
\usepackage{listings}
\usepackage{color}
\usepackage[a4paper, total={6in, 10in}]{geometry}
\usepackage{amsmath}

\definecolor{my_red}{RGB}{216,60,104}
\definecolor{my_green}{RGB}{68,189,77}
\definecolor{my_grey}{RGB}{78,90,107}
\titleformat{\section}{\large\bfseries}{}{0em}{}
\titleformat{\subsection}{\bfseries}{}{0em}{}
\titleformat{\subsubsection}{}{}{2em}{}

\lstdefinestyle{customCStyle}{
  language=C,
  numbers=left,
  stepnumber=1,
  breaklines=true,
  showstringspaces=false,
  keywordstyle=\color{my_red},
  stringstyle=\color{my_green},
  commentstyle=\color{my_grey},
  morecomment=[l][\color{magenta}]
}
\lstset{basicstyle=\ttfamily\small,style=customCStyle}

\begin{document}
\title{Aufgabenblatt 6: Funktionen}
\author{Florian Ludewig (Übungsgruppe 2)}
\maketitle
\section{Aufgabe 1 -- Binär- und Dezimalzahlen}
\begin{lstlisting}
#include <math.h>
#include <stdio.h>

int binaryToDecimal(long int binary)
{
  int decimal = 0, exponent = 0;
  while (binary > 0)
  {
    decimal += binary % 10 == 1 ? pow(2, exponent) : 0;
    binary /= 10;
    exponent++;
  }
  return decimal;
}

long int decimalToBinary(int decimal)
{
  long int binary = 0, i = 0;
  while (decimal > 0)
  {
    int remainder = decimal % 2;
    binary += remainder == 1 ? pow(10, i) : 0;
    decimal /= 2;
    i++;
  }
  return binary;
}

int main()
{
  long int binary = 1000011101010011;
  printf("%ld as decimal is: %d\n", binary, binaryToDecimal(binary));
  int decimal = 34643;
  printf("%d as binary is: %ld\n", decimal, decimalToBinary(decimal));
  return 0;
}
\end{lstlisting}
\pagebreak
\section{Aufgabe 2 -- Quadratische Gleichung}
\begin{lstlisting}
#include <stdio.h>
#include <math.h>

int solve(int a, int b, int c, double *x1, double *x2)
{
  double inside_root = b * b - 4 * a * c;
  *x1 = (-b + sqrt(inside_root)) / (2 * a);
  *x2 = (-b - sqrt(inside_root)) / (2 * a);
  if (inside_root > 0)
    return 2;
  if (inside_root == 0)
    return 1;
  /*
    * if (inside_root < 0)
    * ist in dem Fall redundant
    */
  return 0;
}

int main()
{
  double a, b, c;
  scanf("%lf %lf %lf", &a, &b, &c);
  double x1, x2;
  int anzahl = solve(a, b, c, &x1, &x2);
  if (!anzahl)
    printf("keine Loesung");
  else if (anzahl == 1)
    printf("eine Loesung x: %lf", x1);
  else if (anzahl > 1)
    printf("zwei Loesungen x1: %lf x2: %lf", x1, x2);
}
\end{lstlisting}
\section{Aufgabe 3 -- Mathematische Funktionen}
\begin{lstlisting}
#include <math.h>
#include <stdio.h>

double f1(double x, double y)
{
  if (x == y)
    printf("Teilen durch 0 ist nicht erlaubt!\n");
  return (x + y) / (x - y);
}

double f2(double x, double y)
{
  return sqrt((x * x) + (y * y));
}

int main()
{
  double x = 22, y = 42;
  printf("f1(%lf, %lf) = %lf\n", x, y, f1(x, y));
  printf("f2(%lf, %lf) = %lf\n", x, y, f2(x, y));
  return 0;
}
\end{lstlisting}
\end{document}