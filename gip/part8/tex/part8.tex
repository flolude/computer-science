\documentclass{article}
\usepackage[T1]{fontenc}
\usepackage{titlesec}
\usepackage{listings}
\usepackage{color}
\usepackage[a4paper, total={6in, 10in}]{geometry}
\usepackage{amsmath}
\usepackage[document]{ragged2e}
\usepackage[ngerman]{babel}

\definecolor{my_red}{RGB}{216,60,104}
\definecolor{my_green}{RGB}{68,189,77}
\definecolor{my_grey}{RGB}{78,90,107}
\titleformat{\section}{\large\bfseries}{}{0em}{}
\titleformat{\subsection}{\bfseries}{}{0em}{}
\titleformat{\subsubsection}{}{}{2em}{}

\lstdefinestyle{customCStyle}{
  language=C,
  numbers=left,
  stepnumber=1,
  breaklines=true,
  showstringspaces=false,
  keywordstyle=\color{my_red},
  stringstyle=\color{my_green},
  commentstyle=\color{my_grey},
  morecomment=[l][\color{magenta}]
}
\lstset{basicstyle=\ttfamily\small,style=customCStyle}

\begin{document}
\title{Aufgabenblatt 8: Reihungen}
\author{Florian Ludewig (Übungsgruppe 2)}
\maketitle
\section{Aufgabe 1 -- Integer-Reihung}
\begin{lstlisting}
#include<stdio.h>
#include<stdlib.h>
#include<limits.h>

void readIntArray(int a[], int l) {
  for (int i = 0; i < l; i++) {
    scanf("%d", &a[i]);
  }
}

void printIntArray(int a[], int l) {
  printf("(");
  for (int i = 0; i < l; i++) {
    printf(" %d ", a[i]);
  }
  printf(")");
}

int main(void) {
  int number_of_integers = 0;
  printf("Wie viele ganze Zahlen willst du eingeben?\n");
  scanf("%d", &number_of_integers);
  int *a = malloc(number_of_integers * sizeof(int));
  printf("Gebe nun die Zahlen ein\n");
  readIntArray(a, number_of_integers);
  printf("Du hast folgende Zahlen eingeben:\n");
  printIntArray(a, number_of_integers);

  int max = INT_MIN, min = INT_MAX;
  double average = 0;
  for (int i = 0; i < number_of_integers; i++) {
    if (a[i] > max)
      max = a[i];
    if (a[i] < min)
      min = a[i];
    average += (double)a[i] / number_of_integers;
  }

  printf("\nMaximum: %d\n", max);
  printf("Maximum: %d\n", min);
  printf("Mittelwert: %f\n", average);

  return 0;
}
\end{lstlisting}
\pagebreak
\section{Aufgabe 2 -- Pferderennen}
\begin{lstlisting}
#include <stdlib.h>
#include <time.h>
#include <stdio.h>
#include <limits.h>

void horseRace(int n) {
  int *horses = malloc(n * sizeof(int));
  int round = 1, first = 0, second = 0, third = 0;

  while(1) {
    int random = rand() % n;
    horses[random]++;

    if (random != first && random != second && random != third && horses[random] > horses[third]) {
      third = random;
    }
    if (horses[third] > horses[second]) {
      int oldSecond = second;
      second = third;
      third = oldSecond;
    }
    if (horses[second] > horses[first]) {
      int oldFirst = first;
      first = second;
      second = oldFirst;
    }

    printf("\nRunde %d:\n", round);
    printf("Platz 1: Pferd %d (%dm)\n", first + 1, horses[first]);
    printf("Platz 2: Pferd %d (%dm)\n", second + 1, horses[second]);
    printf("Platz 3: Pferd %d (%dm)\n", third + 1, horses[third]);

    if (horses[random] >= 10)
      break;

    round++;
  }
}

int main(void) {
  srand(time(NULL));
  horseRace(100);
  return 0;
}
\end{lstlisting}
\end{document}