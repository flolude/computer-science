\documentclass{article}
\usepackage[T1]{fontenc}
\usepackage{titlesec}
\usepackage{listings}
\usepackage{color}
\usepackage[a4paper, total={6in, 10in}]{geometry}
\usepackage{amsmath}

\definecolor{my_red}{RGB}{216,60,104}
\definecolor{my_green}{RGB}{68,189,77}
\definecolor{my_grey}{RGB}{78,90,107}
\titleformat{\section}{\large\bfseries}{}{0em}{}
\titleformat{\subsection}{\bfseries}{}{0em}{}
\titleformat{\subsubsection}{}{}{2em}{}

\lstdefinestyle{customCStyle}{
  language=C,
  numbers=left,
  stepnumber=1,
  breaklines=true,
  showstringspaces=false,
  keywordstyle=\color{my_red},
  stringstyle=\color{my_green},
  commentstyle=\color{my_grey},
  morecomment=[l][\color{magenta}]
}
\lstset{basicstyle=\ttfamily\small,style=customCStyle}

\begin{document}
\title{Aufgabenblatt 5: Gleitkommazahlen}
\author{Florian Ludewig (Übungsgruppe 2)}
\maketitle
\section{Aufgabe 1 -- Zinseszinstabelle}
\begin{lstlisting}
#include<stdio.h>
#include<math.h>

int main(void) {
  double capital, interest_rate;
  int years;
  printf("Startkapital in EUR: "); scanf("%lf", &capital);
  printf("Zinssatz in Prozent: "); scanf("%lf", &interest_rate);
  printf("Laufzeit in Jahren: "); scanf("%d", &years);

  for (int i = 1; i <= years; i++) {
    double current_capital = capital * pow((1 + (interest_rate / 100)), i);
    printf("Kapital im %d.ten Jahr: %.2lf EUR\n", i, current_capital);
  }
  return 0;
}
\end{lstlisting}
\section{Aufgabe 2 -- Flächeninhalt eines Dreiecks}
\begin{lstlisting}
#include<stdio.h>
#include<math.h>

struct vector2d {
  float x;
  float y;
};

int main(){
  struct vector2d p1, p2, p3;
  printf("Punkt 1 eingeben: "); scanf("%f %f", &p1.x, &p1.y);
  printf("Punkt 2 eingeben: "); scanf("%f %f", &p2.x, &p2.y);
  printf("Punkt 3 eingeben: "); scanf("%f %f", &p3.x, &p3.y);

  struct vector2d a = { p1.x - p2.x, p1.y - p2.y };
  struct vector2d b = { p1.x - p3.x, p1.y - p3.y };

  float area = (a.x * b.y - a.y * b.x) / 2;
  printf("%f Flaecheneinheiten\n", fabs(area));

  return 0;
}
\end{lstlisting}
\pagebreak
\section{Aufgabe 3 -- Trainingsdaten}
\begin{lstlisting}
#include<stdio.h>
#include<math.h>

int main() {
  int jump_counter = 0;
  float biggest, second_biggest, smallest, average;

  printf("Sprungdaten eingeben:\n");
  while (1) {
    float x;
    scanf("%f", &x);
    if (x <= 0)
      break;
    average = ((average * jump_counter) + x) / (jump_counter + 1);
    if (jump_counter == 0) {
      biggest = x;
      smallest = x;
      second_biggest = x;
    }
    if (x > biggest) {
      second_biggest = biggest;
      biggest = x;
    }
    if (x < smallest)
      smallest = x;
    jump_counter++;
  }

  printf("Anzahl Spruenge: %d\n", jump_counter);
  printf("Groesste Weite: %.2f\n", biggest);
  printf("Zweitgroesste Weite: %.2f\n", second_biggest);
  printf("Kleinste Weite: %.2f\n", smallest);
  printf("Mittlere Weite: %.2f\n", average);
  printf("Differenz der groessten und kleinsten Weite: %.2f\n", fabs(biggest - smallest));

  return 0;
}
\end{lstlisting}
\end{document}