\documentclass{article}
\usepackage[utf8]{inputenc}
\usepackage[german]{babel}
\usepackage{graphicx}
\usepackage{amsmath}
\usepackage{mathtools}
\usepackage{amssymb}
\usepackage{tikz}
\usepackage{rotating}
\usepackage{algpseudocode}

\DeclarePairedDelimiter\ceil{\lceil}{\rceil}
\DeclarePairedDelimiter\floor{\lfloor}{\rfloor}

\title{Algorithmen und Datenstrukturen\\3. Übungsserie}
\author{Florian Ludewig (185722)}
\date\today

\begin{document}
\maketitle

\section*{Aufgabe 1}
\subsection*{a)}
\texttt{STOOGE-SORT} funktioniert für Felder mit 0, 1 oder 2 Elementen:
\begin{itemize}
	\item \textbf{0 Elemente}\\
	      Das Feld ist bereits sortiert.
	\item \textbf{1 Element}\\
	      Das Feld ist bereits sortiert.
	\item \textbf{2 Elemente}\\
	      Sei \texttt{A=[x,y]}, dann ist \texttt{i=0} und \texttt{j=1}.
	      Getauscht wird, wenn $y<x$, wodurch das Feld in jedem Fall korrekt sortiert ist.\\
	      Außerdem ist $l=j-i+1=1-0+1=2$ und $2 \not > 2$ wodurch der Algorithmus terminiert.  
\end{itemize}
Angenommen \texttt{STOOGE-SORT} funktioniert für alle Felder der Länge $n-1$ oder kleiner.\\
Betrachten eines Feldes \texttt{A} der Länge $n$. Der Algorithmus arbeitet drei Schritte ab:
\begin{enumerate}
	\item \texttt{STOOGE-SORT(i,j-k)}\\
	      Hier wird \texttt{STOOGE-SORT} rekursiv mit $i=0$ und $j=(n-1)-\floor{\frac{n}{3}}$ aufgerufen.
	      Weil wir davon ausgehen, dass der Algorithmus für Felder mit kleineren Längen funktioniert,
	      sind nach diesem rekursiven Aufruf alle Zahlen im ersten Drittel des Feldes $[0,...,\floor{\frac{n}{3}}]$
	      kleiner als im zweiten Drittel $[\ceil{\frac{n}{3}},...,n-\floor{\frac{n}{3}}]$.
	\item \texttt{STOOGE-SORT(i+k,j)}\\
	      Nun wird \texttt{STOOGE-SORT} rekursiv mit $i=\floor{\frac{n}{3}}$ und $j=n-1$ aufgerufen.
	      Danach sind mit Sicherheit alle Zahlen im zweiten Drittel des Feldes
	      $[\floor{\frac{n}{3}},...,n-\floor{\frac{n}{3}}]$ kleiner als die Zahlen im letzten Drittel des
	      Feldes $[n-\ceil{\frac{n}{3}},...,n-1]$.\\
	      Demnach befinden sich an diesem Punkt die größten Zahlen im letzten Drittel des Feldes.
	\item \texttt{STOOGE-SORT(i,j-k)}\\
	      Zum Schluss wird \texttt{STOOGE-SORT} nochmal rekursiv mit $i=0$ und $j=(n-1)-\floor{\frac{n}{3}}$
	      aufgerufen, wodurch das Feld in den ersten beiden Dritteln sortiert wird. Weil die größten Zahlen
	      sich bereits im letzten Drittel befinden ist das Feld nach diesem Schritt vollständig sortiert.
\end{enumerate}
\vspace*{1\baselineskip}

\subsection*{b)}
\texttt{STOOGE-SORT} ist ein rekursiver Algorithmus und seine Laufzeit kann durch folgendes rekursives
Schema beschrieben werden:
\[
	T(0)=1,T(1)=1,T(2)=1\\
\]
\[
	T(n)=3\cdot T(\frac{2}{3}n)+O(1)=3\cdot T(\frac{n}{1.5})+O(1)
\]
Weil der Algorithmus sich selbst \textbf{drei} mal Aufruft und jedes mal \textbf{zwei Drittel} des
Feldes sortiert. Mithilfe des ersten Falles des Mastertheorems lässt sich dann die Laufzeit bestimmen:
\[
	T(n)\in \Theta(n^{\log_b a})=\Theta(n^{\log_{\frac{3}{2}} 3}) \approx  \Theta(n^{2.7})
\]

\vspace*{2\baselineskip}
\section*{Aufgabe 2}

\subsection*{a)}
A = (12, 9, 6, 1, 41, 4, 16, 18, 23, 15, 3, 21, 14, 8) als Heap:

\begin{tikzpicture}
	\node{12}
	child { node {9} 
		child { node {1}
			child { node {18} }
			child { node {23} }
		}
		child [ missing ]
		child { node {41}   
			child { node {15} }
			child { node {3} }
		}
	} 
	child [ missing ]
	child [ missing ]
	child [ missing ]
	child {node {6} 
		child { node {4}
			child { node {21} }
			child { node {14} }
		}
		child [ missing ]
		child { node {16} 
			child { node {8} }
		}
	};
\end{tikzpicture}

\begin{itemize}
	\item \textbf{HEAPIFY(A, 7)}\\
	      \begin{tikzpicture}
	      	\node{16}
	      	child { node {8} }
	      	child [ missing ];
	      \end{tikzpicture}
	\item \textbf{HEAPIFY(A, 6)}\\
	      \begin{tikzpicture}[remember picture]
	      	\node(top_left){4}
	      	child { node {21} }
	      	child { node (bottom_left){14} };
	      \end{tikzpicture}
	      $\hphantom{\rightsquigarrow}$
	      \begin{tikzpicture}[remember picture]
	      	\node(top_right){21}
	      	child { node (bottom_right) {4} }
	      	child { node {14} };
	      	\path[overlay] (top_left) --  node(aux1){\vphantom{x}}(bottom_left)
	      	(top_right) --  node(aux2){\vphantom{x}} (bottom_right)
	      	(aux1) -- node{$\rightsquigarrow$}(aux2);
	      \end{tikzpicture}
	\item \textbf{HEAPIFY(A, 5)}\\
	      \begin{tikzpicture}
	      	\node{41}
	      	child { node {15} }
	      	child { node {3} };
	      \end{tikzpicture}
	\item \textbf{HEAPIFY(A, 4)}\\
	      \begin{tikzpicture}[remember picture]
	      	\node(top_left){1}
	      	child { node {18} }
	      	child { node (bottom_left){23} };
	      \end{tikzpicture}
	      $\hphantom{\rightsquigarrow}$
	      \begin{tikzpicture}[remember picture]
	      	\node(top_right){23}
	      	child { node (bottom_right) {18} }
	      	child { node {1} };
	      	\path[overlay] (top_left) --  node(aux1){\vphantom{x}}(bottom_left)
	      	(top_right) --  node(aux2){\vphantom{x}} (bottom_right)
	      	(aux1) -- node{$\rightsquigarrow$}(aux2);
	      \end{tikzpicture}
	\item \textbf{HEAPIFY(A, 3)}\\
	      \begin{tikzpicture}[remember picture]
	      	\node(top_left){6}
	      	child { node {21}
	      		child { node {4} }
	      		child { node {14} }
	      	}
	      	child [ missing ]
	      	child { node {16}
	      		child { node (bottom_left){8} }
	      		child [ missing ]
	      	};
	      \end{tikzpicture}
	      $\hphantom{\rightsquigarrow}$
	      \begin{tikzpicture}[remember picture]
	      	\node(top_right){21}
	      	child { node {14}
	      		child { node (bottom_right){4} }
	      		child { node {6} }
	      	}
	      	child [ missing ]
	      	child { node {16}
	      		child { node {8} }
	      		child [ missing ]
	      	};
	      	\path[overlay] (top_left) --  node(aux1){\vphantom{x}}(bottom_left)
	      	(top_right) --  node(aux2){\vphantom{x}} (bottom_right)
	      	(aux1) -- node{$\rightsquigarrow$}(aux2);
	      \end{tikzpicture}
	\item \textbf{HEAPIFY(A, 2)}\\
	      \begin{tikzpicture}[remember picture]
	      	\node(top_left){9}
	      	child { node {23}
	      		child { node {18} }
	      		child { node {1} }
	      	}
	      	child [ missing ]
	      	child { node {41}
	      		child { node (bottom_left){15} }
	      		child { node {3} }
	      	};
	      \end{tikzpicture}
	      $\hphantom{\rightsquigarrow}$
	      \begin{tikzpicture}[remember picture]
	      	\node(top_right){41}
	      	child { node {23}
	      		child { node (bottom_right){18} }
	      		child { node {1} }
	      	}
	      	child [ missing ]
	      	child { node {15}
	      		child { node {9} }
	      		child { node {3} }
	      	};
	      	\path[overlay] (top_left) --  node(aux1){\vphantom{x}}(bottom_left)
	      	(top_right) --  node(aux2){\vphantom{x}} (bottom_right)
	      	(aux1) -- node{$\rightsquigarrow$}(aux2);
	      \end{tikzpicture}
	\item \textbf{HEAPIFY(A, 1)}\\
	      \begin{tikzpicture}
	      	\node{12}
	      	child { node {41} 
	      		child { node {23}
	      			child { node {18} }
	      			child { node {1} }
	      		}
	      		child [ missing ]
	      		child { node {15}   
	      			child { node {9} }
	      			child { node {3} }
	      		}
	      	} 
	      	child [ missing ]
	      	child [ missing ]
	      	child [ missing ]
	      	child {node {21} 
	      		child { node {14}
	      			child { node {4} }
	      			child { node {6} }
	      		}
	      		child [ missing ]
	      		child { node {16} 
	      			child { node {8} }
	      			child [ missing ]
	      		}
	      	};
	      \end{tikzpicture}	              
	      \begin{center}
	      	\begin{rotate}{-90}$\rightsquigarrow$\end{rotate}
	      \end{center}
	      \vspace*{1\baselineskip}
	      \begin{tikzpicture}
	      	\node{41}
	      	child { node {23} 
	      		child { node {18}
	      			child { node {12} }
	      			child { node {1} }
	      		}
	      		child [ missing ]
	      		child { node {15}   
	      			child { node {9} }
	      			child { node {3} }
	      		}
	      	} 
	      	child [ missing ]
	      	child [ missing ]
	      	child [ missing ]
	      	child {node {21} 
	      		child { node {14}
	      			child { node {4} }
	      			child { node {6} }
	      		}
	      		child [ missing ]
	      		child { node {16} 
	      			child { node {8} }
	      			child [ missing ]
	      		}
	      	};
	      \end{tikzpicture}
\end{itemize}
\vspace*{1\baselineskip}

\subsection*{b)}
\begin{itemize}
	\item \textbf{Nach 1x \texttt{Heap-Extract-Max(A)}}\\
	      \begin{tikzpicture}
	      	\node{23}
	      	child { node {18} 
	      		child { node {12}
	      			child { node {8} }
	      			child { node {1} }
	      		}
	      		child [ missing ]
	      		child { node {15}   
	      			child { node {9} }
	      			child { node {3} }
	      		}
	      	} 
	      	child [ missing ]
	      	child [ missing ]
	      	child [ missing ]
	      	child {node {21} 
	      		child { node {14}
	      			child { node {4} }
	      			child { node {6} }
	      		}
	      		child [ missing ]
	      		child { node {16}	}
	      	};
	      \end{tikzpicture}
	\item \textbf{Nach 2x \texttt{Heap-Extract-Max(A)}}\\
	      \begin{tikzpicture}
	      	\node{21}
	      	child { node {18} 
	      		child { node {12}
	      			child { node {8} }
	      			child { node {1} }
	      		}
	      		child [ missing ]
	      		child { node {15}   
	      			child { node {9} }
	      			child { node {3} }
	      		}
	      	} 
	      	child [ missing ]
	      	child [ missing ]
	      	child [ missing ]
	      	child {node {16} 
	      		child { node {14}
	      			child { node {4} }
	      			child [ missing ]
	      		}
	      		child [ missing ]
	      		child { node {6}	}
	      	};
	      \end{tikzpicture}
	\item \textbf{Nach 3x \texttt{Heap-Extract-Max(A)}}\\
	      \begin{tikzpicture}
	      	\node{21}
	      	child { node {18} 
	      		child { node {12}
	      			child { node {8} }
	      			child { node {1} }
	      		}
	      		child [ missing ]
	      		child { node {15}   
	      			child { node {9} }
	      			child { node {3} }
	      		}
	      	} 
	      	child [ missing ]
	      	child [ missing ]
	      	child [ missing ]
	      	child {node {16} 
	      		child { node {14} }
	      		child [ missing ]
	      		child { node {6}	}
	      	};
	      \end{tikzpicture}
\end{itemize}

\vspace*{1\baselineskip}

\vspace*{2\baselineskip}
\section*{Aufgabe 3}
\subsection*{a)}
\begin{itemize}
	\item \texttt{HeapDelete(A,i)}
	      \begin{algorithmic}
	      	\State $A[i] \Leftarrow A[\text{heapsize}[A]]$
	      	\State $\text{heapsize}[A] \Leftarrow \text{heapsize}[A] - 1$
	      	\State $\text{MaxHeapify}(A,i)$
	      \end{algorithmic}
	      \texttt{MaxHeapify} hat eine Laufzeit von $\text{O}(\text{log }n)$. Somit hat dieser \texttt{HeapDelete}
	      Algorithmus ebenfalls eine Laufzeit von $\text{O}(\text{log }n)$.
\end{itemize}  
\vspace*{1\baselineskip}

\subsection*{b)}
\begin{itemize}
	\item \texttt{FindMin(A,i)}
	      \begin{algorithmic}
	      	\State $min \Leftarrow A[\floor{\frac{n}{2}}+1]$
	      	\For{$i \Leftarrow \floor{\frac{n}{2}}+2$ \textbf{to} $n$}
	      	\If{$A[i] < min$}
	      	\State{$min \Leftarrow A[i]$}
	      	\EndIf
	      	\EndFor
	      	\State \textbf{return} $min$
	      \end{algorithmic}
	      \texttt{FindMin} brauchst $n-\floor{\frac{n}{2}}+1$ Schritte und hat somit eine Laufzeit von $\text{O}(n)$.
\end{itemize}  

\vspace*{2\baselineskip}
\section*{Aufgabe 4}

\subsection*{a)}
\[ T(n)=5\cdot T(\frac{n}{2})+n\sqrt{n} \rightsquigarrow a=5, b=2, f(n)=n\sqrt{n} \]
\[ n^{log_b a}=n^{log_2 5} \approx n^{2.3} > n^{1.5}=n\cdot n^{0.5} = f(n) \]
\[ \text{Fall 1, weil } f(n) \in \text{O} (n^{log_b(a) - \epsilon}) \approx \text{O}(n^{2.3 - \epsilon}) \]
\[ \implies T(n) \in \Theta (n^{log_2 5}) \]
\vspace*{1\baselineskip}

\subsection*{b)}
\[ T(n)=16\cdot T(\frac{n}{8})+n^{\frac{4}{3}} \rightsquigarrow a=16, b=8, f(n)=n^\frac{4}{3} \]
\[ n^{log_b a}=n^{log_8 16} = n^\frac{4}{3} = f(n) \]
\[ \text{Fall 2, weil } f(n) \in \Theta (n^{log_b a}) = \Theta (n^\frac{4}{3}) \]
\[ \implies T(n) \in \Theta (n^\frac{4}{3} \cdot \text{log }n) \]
\vspace*{1\baselineskip}

\subsection*{c)}
\[ T(n)=5\cdot T(\frac{n}{7})+n\text{ log }n \rightsquigarrow a=5, b=7, f(n)=n\text{ log }n \]
\[ n^{log_b a}=n^{log_7 5} \approx n^{0.8} < n^1 < f(n) \]
\[ \text{Fall 3, weil } f(n) \in \Omega (n^{log_b a + \epsilon}) \approx \Omega (n^{0.8 + \epsilon}) \]
\[ \text{Regularitätsbedingung prüfen:} \]
\[ a\cdot f(\frac{n}{b}) \leq c \cdot f(n) \Leftrightarrow \]
\vspace*{1\baselineskip}

\subsection*{d)}
\[ T(n)=2\cdot T(\frac{n}{4})+3n^{\frac{1}{4}}+\text{ log }n \rightsquigarrow a=2, b=4, f(n)=3n^{\frac{1}{4}}+\text{ log }n \]
\[ n^{log_b a}=n^{log_4 2} = n^\frac{1}{4} < 3n^{\frac{1}{4}} < f(n) \]
\vspace*{1\baselineskip}

\subsection*{e)}
\[ T(n)=16\cdot T(\frac{n}{4})+n^2\text{ log }n \rightsquigarrow a=16, b=4, f(n)=n^2\text{ log }n \]
\[ n^{log_b a}=n^{log_4 16} = n^2 < f(n) \]

\end{document}